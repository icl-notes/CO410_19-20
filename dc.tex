\chapter{Data Center and cloud}

\paragraph{Cloud}. Large pool of easily usable \textbf{virtualized} computing
resources, development \textbf{platforms} and various \textbf{services} and
application (metered service over a network).
\begin{multicols}{2}
\begin{itemize}
\item[$+$] Speed – services provided on demand
\item[$+$] Global scale and elasticity
\item[$+$] Productivity
\item[$+$] Performance and Security
\item[$+$] Customizability
\item[$-$] Dependency on network and internet connectivity
\item[$-$] Security and Privacy
\item[$-$] Cost of migration
\item[$-$] Cost and risk of vendor lock-in
\end{itemize}
\end{multicols}

\paragraph{Types of cloud}
\begin{itemize}
\item \textbf{Public}: Cloud vendors offer their computing resources over the Internet
\item \textbf{Private}: infrastructure used exclusively by a single business. Services
  on private network
\item \textbf{Hybrid}: data and applications are shared between the above. Easier for
  optimizing existing infrastructure, security and compliance
\end{itemize}

\paragraph{Cloud Service Models}
\begin{itemize}
  \item \textbf{IaaS}: IT infrastructure – servers and VMs, storage, networks, firewall and security. Resources pooled together for multiple users. User given most control.
  \item \textbf{PaaS}: Environment for developing, testing and managing applications – servers, storage, network, OS, middleware, databases
  \item \textbf{FaaS}: (Serverless) Vendor does set-up, capacity planning, and server management, you just supply the logic/functions.
  \item \textbf{SaaS}: Software over the Internet. Entirely managed by cloud
    provider (updates, patches)
\end{itemize}

\paragraph{Data centers} Physical facility that enterprises use to house
computing and infrastructure in a variety of networked formats.
\begin{itemize}
\item PCSS: Power, Cooling, Shelter, Security
\item (Total Cost of Ownership) TCO = CapEx + OpEx (capital + operations). Cloud removes CapEx.
\item Power Usage Effectiveness: PUE = total energy used (all overhead) : energy delivered to the computing equipment (needed)
\item DCs consume 3\% global electricity, produce 2\% total greenhouse emissions. Monthly costs = \$3,530,920
\item Reduce costs: good location for cooling and power load factor (arctic, dam), raise temp (more failures but less cooling costs), reuse heat, 
  reduce conversion of energy (e.g. work at higher voltage)
\item Energy consumption not proportional to load - try to optimise workloads     so that the servers aren't idle. Need resource virtualisation +
  pooling to consolidate service on fewer servers. 6-15\% utilisation
  to 30\% (virtualised).
\item CPUs usually idle because: DC providers want to ensure low latency even at the tail ends; provisioning servers for peak demand; overhead for e.g. changing tenants; I/O rather than CPU bound programs
\item Cloud providers must treat VMs as black boxes. Tenants may overallocate resources accidentaly - then vendor may want to utilise those unused resources. But tenants may deliberately overallocate to get really low latency. 
\end{itemize}

\paragraph{Improving Resource Utilisation}
\begin{itemize}
\item Hyperscale system management software: DC as a warehouse-scale computer.
  Software managed pooled resources that include compute, network, and
  storage.
\item Dynamic resource allocation: Virtualisation isn't enough, need to
  dynamically allocate CPU resources across servers and racks, allowing admins
  to quickly migrate resources to address the shifting demand (2x to 6x better).
\item Networking: Software Defined Networking (SDN). Improve internal
  machine-to-machine communication with custom protocols.
\end{itemize}

\paragraph{Rack scale computing} Made of compute (\& accelerators), storage
(hot / warm / cold disks), networking (interconnect, SDN).
\begin{itemize}
\item Evolution - server-centric to resource-centric design: 
  \begin{itemize}
    \item Before: physical aggregation (shared power, cooling, rack-management)
    \item Now: fabric integration (fast rack-wide interconnect)
    \item Goal: resource disaggregation (pooled compute, storage, memory
  resources). Benefits: Scale-out by default, fine-grained control over parts (e.g. can replace one failing thing), finer resource allocation (give application exactly whats needed), better at stopping interference between workloads (less tenant interaction).
  \end{itemize}
\item Future: Heterogeneous Computing Resources across the Rack: Accelerators,
  Co-processors, Intelligent storage, Intelligent (active) memory, Smart NICs,
  In-network data processing
\end{itemize}
