\chapter{Data Center and cloud}

\paragraph{Cloud}. Large pool of easily usable \textbf{virtualized} computing
resources, development \textbf{platforms} and various \textbf{services} and
application (metered service over a network).
\begin{multicols}{2}
\begin{itemize}
\item[$+$] Speed – services provided on demand
\item[$+$] Global scale and elasticity
\item[$+$] Productivity
\item[$+$] Performance and Security
\item[$+$] Customizability
\item[$-$] Dependency on network and internet connectivity
\item[$-$] Security and Privacy
\item[$-$] Cost of migration
\item[$-$] Cost and risk of vendor lock-in
\end{itemize}
\end{multicols}

\paragraph{Types of cloud}
\begin{itemize}
\item \textbf{Public}: Cloud vendors offer their computing resources over the Internet
\item \textbf{Private}: infrastructure used exclusively by a single business. Services
  on private network
\item \textbf{Hybrid}: data and applications are shared between the above. Easier for
  optimizing existing infrastructure, security and compliance
\end{itemize}

\paragraph{Cloud Service Models}. google app engine
\begin{itemize}
  \item \textbf{IaaS}: IT infrastructure – servers and VMs, storage, networks, firewall and security
  \item \textbf{PaaS}: Environment for applications – servers, storage, network,
    OS, middleware, databases
  \item \textbf{FaaS}: Set-up, capacity planning, and server management
  \item \textbf{SaaS}: Software over the Internet. Entirely managed by cloud
    provider (updates, patches)
\end{itemize}

\paragraph{Data centers} Physical facility that enterprises use to house
computing and infrastructure in a variety of networked formats.
\begin{itemize}
\item PCSS: Power, Cooling, Shelter, Security
\item TCO = CapEx + OpEx (capital + operations)
\item Power Usage Effectiveness: PUE = ${E_{used}} / {E_{given}}$
\item 3\% electricity, 2\% Greenhouse. Monthly costs = \$3,530,920
\item Reduce costs: raise temp, reuse heat, choose location (arctic, dam),
  reduce conversion of $E$
\item Energy consumption not proportional to load. Need Virtualisation +
  resource pooling to consolidate service on fewer servers. 6-15\% utilisation
  to 30\% (virtualised).
\end{itemize}

\paragraph{Improving Resource Utilisation}
\begin{itemize}
\item Hyperscale system management software: DC as a warehouse-scale computer.
  Software managed pooled resources that include compute, network, and
  storage.
\item Dynamic resource allocation: Virtualisation isn't enough, need to
  dynamically allocate CPU resources across servers and racks, allowing admins
  to quickly migrate resources to address the shifting demand (2x to 6x better).
\item Networking: Software Defined Networking (SDN). Improve internal
  machine-to-machine communication with custom protocols.
\end{itemize}

\paragraph{Rack scale computing} Made of Compute (\& accelerators), Storage
(hot / warm / cold disks), Networking (interconnect, SDN).
\begin{itemize}
\item evolution: physical aggregation (shared power, cooling, rack-management)
  $\rightarrow$ : fabric integration (fast rack-wide interconnect)
  $\rightarrow$ resource disaggregation (pooled compute, storage, memory
  resources)
\item Heterogeneous Computing Resources across the Rack: Accelerators,
  Co-processors, Intelligent storage, Intelligent (active) memory, Smart NICs,
  In-network data processing
\end{itemize}
